\usepackage{amsmath} % Nice maths symbols.
\usepackage{amssymb} % Nice variable symbols.
\usepackage{array} % Allow for custom column widths in tables.
\usepackage{arydshln} % Dashed lines using \hdashline \cdashline
\usepackage{bbm} % Gives Blackboard fonts.
\usepackage{bm} % Bold math symbols.
\usepackage[margin=0pt,font=small,labelfont=bf,labelsep=endash, justification=justified]{caption} % Caption figures and tables nicely.\\
\usepackage{esdiff} % Gives nice differential operators.
\usepackage{esvect} % Gives nice vector arrows.
\usepackage{float} % Nice figure placement.
\usepackage{graphicx} % Include figures.
\usepackage{indentfirst} % Indents the first paragraph.
\usepackage{letltxmacro} % For defining a nice SQRT symbol.
\usepackage{lipsum} % Useful for block text.
%\usepackage{listings} % The listings package for code.
\usepackage[framed,numbered,autolinebreaks,useliterate]{mcode} % Inports Listings package ideal for MATLAB.
\usepackage{multirow} % Nice table cells spanning many rows.
\usepackage{multicol} % If I want to use multiple columns.
\usepackage[numbers, sort&compress]{natbib} % Nice references.
\usepackage{physics} % Nice partial derivatives and BRAKET notation.
\usepackage{ragged2e}  % Nicer justification.
\usepackage{subcaption} % Side by side figures.
\usepackage{tikz} % Nice diagrams.
\usepackage{xspace} % Gives nice spacing for commands.
% Generally HYPERREF should be imported last.
\usepackage{hyperref} % Colour links. % Should be used near the end.
\usepackage{geometry} % Use nice margins. % Should be loaded after hyperref.

% maming nice link colors.
\hypersetup{colorlinks,linkcolor=blue,urlcolor=blue,citecolor=blue,anchorcolor=blue} 


% If you want to consider serif fonts...
%\usefonttheme{serif} % To make everything serif.
\usefonttheme[onlymath]{serif} % To make the maths serif.

% To make beamer justified.
\justifying
\addtobeamertemplate{block begin}{}{\justifying}

% Use numbered figures.
\setbeamertemplate{caption}[numbered]


% Present the references in the order they are used.
\bibliographystyle{unsrtnat}

% Listing -> Code in environment labels.
\renewcommand{\lstlistingname}{Code}


% Giving the refereneces the right title.
\renewcommand{\bibname}{References}

% Gives the nicer SQRT symbol.
\usepackage{letltxmacro} 
\makeatletter
\let\oldr@@t\r@@t
\def\r@@t#1#2{%
	\setbox0=\hbox{$\oldr@@t#1{#2\,}$}\dimen0=\ht0
	\advance\dimen0-0.2\ht0
	\setbox2=\hbox{\vrule height\ht0 depth -\dimen0}%
	{\box0\lower0.4pt\box2}}
\LetLtxMacro{\oldsqrt}{\sqrt}
\renewcommand*{\sqrt}[2][\ ]{\oldsqrt[#1]{#2}}
\makeatother

% Makes beamer lists nicely justified.
\makeatletter
\renewcommand{\itemize}[1][]{%
	\beamer@ifempty{#1}{}{\def\beamer@defaultospec{#1}}%
	\ifnum \@itemdepth >2\relax\@toodeep\else
	\advance\@itemdepth\@ne
	\beamer@computepref\@itemdepth% sets \beameritemnestingprefix
	\usebeamerfont{itemize/enumerate \beameritemnestingprefix body}%
	\usebeamercolor[fg]{itemize/enumerate \beameritemnestingprefix body}%
	\usebeamertemplate{itemize/enumerate \beameritemnestingprefix body begin}%
	\list
	{\usebeamertemplate{itemize \beameritemnestingprefix item}}
	{\def\makelabel##1{%
			{%
				\hss\llap{{%
						\usebeamerfont*{itemize \beameritemnestingprefix item}%
						\usebeamercolor[fg]{itemize \beameritemnestingprefix item}##1}}%
			}%
		}%
	}
	\fi%
	\beamer@cramped%
	\justifying% NEW
	%\raggedright% ORIGINAL
	\beamer@firstlineitemizeunskip%
}
\makeatother


% Removes hyphenation
\tolerance=1
%\emergencystretch=\maxdimen
\hyphenpenalty=10000
\hbadness=10000

% Custom column widths using C{2cm}, L, R, etc.
\newcolumntype{L}[1]{>{\raggedright\let\newline\\\arraybackslash\hspace{0pt}}m{#1}}
\newcolumntype{C}[1]{>{\centering\let\newline\\\arraybackslash\hspace{0pt}}m{#1}}
\newcolumntype{R}[1]{>{\raggedleft\let\newline\\\arraybackslash\hspace{0pt}}m{#1}}
